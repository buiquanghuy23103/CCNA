\chapter{Introduction}

\section{Securing network}

\subsection{Attack vector}

An attack vector is a path or other means by which an attacker can gain access to a server, host, or network. Many attack vectors originate from outside the corporate network. For example, attackers may target a network, through the Internet, in an attempt to disrupt network operations and create a denial of service (DoS) attack.\\

Attack vectors can also originate from inside the network. An internal user, such as an employee, can accidently or intentionally steal and copy confidential data to removable media. Internal threats also have the potential to cause greater damage than external threats because internal users have direct access to the building and its infrastructure devices. Employees also have knowledge of the corporate network, its resources, and its confidential data.\\

Network security professionals must implement tools and apply techniques for mitigating both external and internal threats.

\subsection{Data loss}

Various Data Loss Prevention (DLP) controls must be implemented to protect the organization's data. Common data loss vectors are shown in the following list:

\begin{itemize}
\item \textbf{Email/Social networking:} instant messaging software and social media sites
\item \textbf{Unencrypted devices:} a stolen laptop typically contains confidential data. If the information is not stored using encryption algorithm, the attacker can retrieve valuable confidential data.
\item \textbf{Cloud storage service:} sensitive data can be loss if access to cloud service is compromised.
\item \textbf{USB storage:} employees may perform unauthorized data transfer to USB; data stored in USB devices can be lost due to physical damage
\item \textbf{Hard copy:} confidential data on paper should be shredded when no longer required
\item \textbf{Improper access control:} stolen passwords or weak passwords
\end{itemize}

\subsection{Network topology}

\paragraph{Campus Area Networks:} The main focus of this course is on securing Campus Area Networks (CANs). CANs consists of interconnected LANs within a limited geographic area. Figure \ref{CANsec} displays a sample CAN with a defense in-depth approach using various security features and security devices to secure it.

\begin{figure}[hbtp]
\caption{CAN security}\label{CANsec}
\centering
\includegraphics[scale=0.5]{pictures/CANsec.PNG}
\end{figure}

\paragraph{SOHO network:} Attackers may want to use someone's Internet connection for free, use the Internet connection for illegal activity, or view financial transactions, such as online purchases. Home networks and SOHOs are typically protected using a consumer \emph{grade router}, such as a \emph{Linksys home wireless router}. 

\paragraph{WAN network:} Network security professionals must use secure devices on the edge of the networks. The Main site and Regional site are protected by an ASA, which provides stateful firewall features and establishes VPN tunnels to various destinations. The branch site is secured using hardened ISR, which can establish an always-on VPN connection to the main site. The SOHO and Mobile users connect to the main site using Cisco Anyconnect VPN client. 

\paragraph{Data center network:} Data center networks are interconnected to corporate sites using \emph{VPN} technology with \emph{ASA} devices and \emph{integrated data center switches}, such as a high-speed Nexus switches. Data center physical security can be divided into two areas: Outside perimeter security and Inside perimeter security. 
%\emph{Security traps} (Inside perimeter security) provide access to the data halls where data center data is stored. A person must first enter the security trap using their badge ID proximity card. After the person is inside the security trap, biometric verifications are used to open the second door. The user must repeat the process to exit the data hall.

\paragraph{Cloud and virtual network:} This kind of network uses virtual machines (VM) to provide services to their clients.  VMs are also prone to specific targeted attacks as shown in the following list. The Cisco Secure Data Center is a solution to secure Cloud and virtual network. The core components of this solution provide: Secure Segmentation, Threat Defense, and Visibility.

\begin{itemize}
\item \textbf{Hyperjacking:} An attacker could hijack a VM hypervisor and use it as a starting point to attack other devices.
\item \textbf{Instant on activation:} A VM that has not been used for a long period of time can introduce security vulnerabilities when activated.
\item \textbf{Antivirus storm:} Multiple VMs attempt to download antivirus file at the same time
\end{itemize}

\paragraph{Borderless Network:} To accommodate the BYOD trend, Cisco developed the Borderless Network. To support this network , Cisco devices support Mobile Device Management (MDM) features. MDM features secure, monitor, and manage mobile devices, including corporate-owned devices and employee-owned devices. Some critical MDM functions include:

\begin{itemize}
\item Data encryption
\item PIN enforcement
\item Data wipe (lost devices can be remotely wiped out)
\item Data Loss Prevention (DLP): prevent unauthorized users from accessing, prevent authorized users from doing malicious things to  critical data  
\item Detect password bypasses such as Jalibreaking (Apple iOS) or Rooting (Android) and restrict devices' access to network and corporate assets
\end{itemize}

\section{Network threats}

\subsection{Who is a hacker?}

Hacker is a common term used to describe a network attacker. The terms white hat hacker, black hat hacker, and grey hat hacker are often used to describe hackers. The \emph{white hat hackers} are ethical hackers that perform network penetration test to discover network vulnerabilities. The \emph{grey hat hackers} do unethical things but not for personal gain or to cause damage (e.g. disclose vulnerability publicly). The \emph{black hat hackers} violate computer and network security for personal gain and malicious purposes. \\

The following list displays modern hacking terms and a brief description of each.\\

\begin{itemize}
\item Script kiddies -- inexperienced hackers running scripts, tools, programs, etc. to cause harm but not for profit
\item Vulnerability broker -- white hat hackers discover exploits for reward
\item Hacktivists -- protest against political or social ideas by leaking sensitive information
\item Cyber criminals -- black hat hackers
\item State-sponsored -- either white hat or black hat hacker, who steals government secrets, sabotage network, and intelligence; their targets are foreign government, terrorist group, and corporations
\end{itemize}

%\subsection{Hacker tools}
%
%\begin{itemize}
%\item Password crackers -- repeatedly make guesses to crack the password; sometimes referred to as password recovery tool
%\item Wireless hacking tool
%\item Network scanning and hacking tools -- probe network devices for open TCP or UDP ports
%\item Packet crafting tool -- probe and test firewall's robustness
%\item Packet sniffer -- capture and analyze packets in Ethernet LAN or WLAN
%\item Rootkit detector -- used by white hat hackers to detect installed root kits
%\item Fuzzer -- discover computer's security vulnerability
%\item Forensic tools -- sniff out any trace of evidence existing
%\item Debuggers -- used by black hat hackers to reverse engineer binary files when writing exploits; used by white hat hackers to analyze malware
%\item Hacking OS -- designed OS preloaded with tools and technologies optimized for hacking, such as Kali Linux, SELinux, Knoppix, Backbox Linux
%\item Encryption tools
%\item Vulnerability scanners exploitation tool
%\item Vulnerability exploitation tool
%\end{itemize}
%
%Hackers can use the previously mentioned attack tools or a combination of tools to create various attacks. The figure displays common types of hacking attacks.
%
%\begin{itemize}
%\item Evesdropping
%\item Data modification
%\item IP address spoofing
%\item Password-based -- If hackers discover a valid account, they could obtain a list of other users, change configurations, etc.
%\item DoS
%\item Man-in-the-Middle: hackers position themselves between the source and destination to monitor, capture, and control communication
%\item Compromised key
%\end{itemize}

\subsection{Malware}

A \textbf{virus} is malicious code that is attached to executable files which are often legitimate programs. Most viruses require end user \emph{activation}. A simple virus may install itself at the first line of code on an executable file. When activated, the virus might check the disk for other executables so that it can infect all the files it has not yet infected. Most viruses are now \emph{spread} by USB memory drives, CDs, DVDs, network shares, and email. Email viruses are now the most common type of virus.\\

A \textbf{Trojan horse} is malware that carries out malicious operations under the guise of a desired function. A Trojan horse comes with malicious code hidden inside of it. This malicious code exploits the \emph{privileges} of the user that \emph{runs} it. Often, Trojans are found attached to online games. \\

\textbf{Worms} \emph{replicate} themselves by independently exploiting vulnerabilities in networks. Worms usually \emph{slow down} networks. Whereas a virus requires a host program to run, worms can \emph{run by themselves}. Other than the initial infection, they no longer require user participation. After a host is infected, the worm is able to \emph{spread very quickly} over the network. Most worm attacks consist of three components:

\begin{itemize}
\item \textbf{Enabling vulnerability:} A worm installs itself using an exploit mechanism, such as an email attachment, an executable file, or a Trojan horse.
\item \textbf{Propagation mechanism:} After gaining access to a device, the worm replicates itself and locates new targets.
\item \textbf{Payload:} Any malicious code that results in some action is a payload. Most often this is used to create a backdoor to the infected host or create a DoS attack.
\end{itemize}

\note Worms never really stop on the Internet. After they are released, they continue to propagate until all possible sources of infection are properly patched.\\

Some examples of modern malware:

\begin{itemize}
\item Ransomware -- deny access to the infected computer system, then demand a paid ransom for the restriction to be removed.
\item Spyware -- gather information about a user and send the information to another entity
\item Adware -- display annoying pop-up advertising pertinent to websites visited
\item Scareware -- include scam software which uses social engineering to shock or induce anxiety by creating the perception of a threat
\item Phishing -- attempt to convince people to divulge sensitive information, e.g. receiving an email from their bank asking users to divulge their account and PIN numbers.
\item Rootkits -- installed on a compromised system, then hide its intrusion and maintain privileged access to the hacker.
\end{itemize}

\subsection{Common network attacks}

The method used in this course classifies attacks in three major categories: Reconnaissance, Access, and DoS Attacks.\\

\textbf{Reconnaissance} is known as information gathering. Hackers use reconnaissance (or recon) attacks to do unauthorized discovery and mapping of systems, services, or vulnerabilities. Some examples of reconnaissance attacks:  information query, ping sweep,  port scan,  Vulnerability Scanners,  Exploitation tools.\\

\textbf{Access} attacks exploit known vulnerabilities in authentication services, FTP services, and web services to gain entry to  sensitive information. There are five common types of access attacks: Password attack, Trust exploitation, Port redirection, Man-in-the-middle, Buffer overflow, IP, MAC, DHCP Spoofing.\\

\textbf{Social engineering} is an \textbf{access attack} that attempts to manipulate individuals into performing actions or divulging confidential information. Specific types of social engineering attacks include:

\begin{itemize}
\item Pretexting -- a hacker calls an individual and lies to them in an attempt to gain access to privileged data
\item Phishing -- a malicious party sends a fraudulent email disguised as being from a legitimate, trusted source. The message intends to trick the recipient into installing malware on their device, or into sharing personal or financial information.
\item Spam -- use spam email to trick a user to click an infected link or download an infected file.
\item Tailgating -- This is when a hacker quickly follows an authorized person into a secure location. The hacker then has access to a secure area.
\item Something for Something (Quid pro quo) -- a hacker requests personal information from a party in exchange for something like a free gift.
\item Baiting -- a hacker leaves a malware-infected physical device, such as a USB flash drive in a public location such as a corporate washroom. The finder finds the device and loads it onto their computer, unintentionally installing the malware.
\end{itemize}

\textbf{Denial-of-Service (DoS)} attacks are highly publicized network attacks and relatively simple to conduct. A DoS attack results in some sort of interruption of service to users, devices, or applications. There are two major sources of DoS attacks:

\begin{itemize}
\item Maliciously Formatted Packets -- a maliciously formatted packet is forwarded to a host and the receiver is unable to handle an unexpected condition. This causes the receiving device to crash or run very slowly.
\item Overwhelming Quantity of Traffic -- This is when a network, host, or application is unable to handle an enormous quantity of data, causing the system to crash or become extremely slow.
\end{itemize}

\textbf{A Distributed DoS Attack (DDoS)} is similar in intent to a DoS attack, except that a DDoS attack increases in magnitude because it originates from multiple, coordinated sources. As an example, a DDoS attack could proceed as follows:

\begin{enumerate}
\item A hacker builds a network of infected machines. A network of infected hosts is called a \emph{botnet}. The compromised computers are called \emph{zombie computers}, and they are controlled by \emph{handler systems}.
\item The zombie computers continue to scan and infect more targets to create more zombies.
\item When ready, the hacker instructs the handler systems to make the botnet of zombies carry out the DDoS attack.
\end{enumerate}

\section{Mitigating Threats}

\subsection{Defending networks}

\textbf{Cryptography}, the study and practice of hiding information, is used extensively in modern network security. Cryptography ensures three components of information security: Confidentiality, Integrity, and Availability.\\

The 12 \textbf{network security domains} are intended to serve as a common basis for developing organizational security standards and effective security management practices. They provide a convenient separation of the elements of network security. One of the most important domains is the \textbf{security policy domain}. A security policy is a formal statement of the rules by which people that are given access to the technology and information assets of an organization, must abide. \\

\textbf{Security Artichoke} is the analogy used to describe what a hacker must do to launch an attack in a Borderless network. They  remove certain \emph{artichoke leafs}, and each \emph{leaf} of the network may reveal some sensitive data. And leaf after leaf, it all leads the hacker to more data. 

\subsection{Cisco SecureX architecture}

The Cisco SecureX architecture is designed to provide effective security for any user, using any device, from any location, and at any time. This architecture includes the following five major components:

\begin{itemize}
\item \textbf{Scanning Engines:} the workhorses of policy enforcement; examine the content, authenticate uses, and identify applications.
\item \textbf{Delivery Mechanisms:} introduce scanning elements into the network; Example: a module in switch or router, an image of Cisco security cloud
\item \textbf{Security Intelligence Operations (SIO):} the brain; they distinguish good traffic from malicious traffic
\item \textbf{Policy Management Consoles:} they are separated from the scanners that enforce the policy, by doing this, it is possible to span a single policy definition to multiple enforcement points
\item \textbf{Next-Generation Endpoints:} critical piece that ties all everything together; they can be any of a multitude of devices; 
\end{itemize}

A \textbf{context-aware scanning element} is a network security device that examines packets on the wire, but also looks at external information to understand the full context of the situation. A context-aware policy uses a simplified descriptive business language to define security policies based on five parameters: person ID, application, device type, location, and access time.\\

To help keep the Cisco ESA, WSA, ASA, and IPS devices secure, Cisco designed the \textbf{Security Intelligence Operations (SIO)}. The SIO is a Cloud-based service that connects global threat information, reputation-based services, and sophisticated analysis, to Cisco network security devices.

\subsection{Mitigating common network attacks}

\paragraph{Malware:} The primary means of mitigating virus and Trojan horse attacks is antivirus software. Antivirus software helps prevent hosts from getting infected and spreading malicious code. Antivirus products are host-based. These products are installed on computers and servers to detect and eliminate viruses. However, they do not prevent viruses from entering the network.

\paragraph{Worms:} They are more network-based than viruses. Worm mitigation requires diligence and coordination on the part of network security professionals. The response to a worm attack can be broken down into four phases:

\begin{enumerate}
\item \textbf{\textbf{Containment}:} limit the spread of worm infection
\item \textbf{Inoculation:} run parallel to or subsequent to the Containment phase; all uninfected systems are patched with the appropriate vendor patch.
\item \textbf{Quarantine:} identify the infected machines
\item \textbf{Treatment:} disinfect the infected systems
\end{enumerate}

\paragraph{Reconnaissance:} We can detect Reconnaissance attacks and generate an alarm using \emph{Anti-sniffer} software and hardware tools, Cisco ASA, Cisco ISR. \emph{Encryption} is an effective solution for sniffer attacks. Using \emph{IPS} and \emph{firewall} can limit the impact of \emph{port scanning}. \emph{Ping sweeps} can be stopped if \emph{ICMP} echo and echo-reply are turned off on edge routers. 

\paragraph{Access attacks:} In general, access attacks can be detected by reviewing logs, bandwidth utilization, and process loads. The use of encrypted or hashed authentication protocols, along with a strong password policy, greatly reduces the probability of successful access attacks. Educate employees about the risks of social engineering, and develop strategies to validate identities.

\paragraph{DoS attacks:} One of the first signs of a DoS attack is a large number of user complaints about unavailable resources. To minimize the number of attacks, a network utilization software package should be running at all times. 

\subsection{Cisco Network Foundation Protection Framework}

The Cisco Network Foundation Protection (NFP) framework provides comprehensive guidelines for protecting the network infrastructure. These guidelines form the foundation for continuous delivery of service. NFP logically divides routers and switches into three functional areas: Control plane, Management plane, and Data plane.\\

\textbf{Control plane} security can be implemented using the following features: \emph{Routing protocol authentication}, \emph{CoPP}, and \emph{AutoSecure}. CoPP (Control Plane Policing) prevents unnecessary traffic from overwhelming the route processor. AutoSecure can lock down the management plane functions and the forwarding plane services and functions of a router.\\

\textbf{Management plane} security can be implemented using the following features:

\begin{itemize}
\item Login and password policy (\verb|enable secret| command)
\item Present legal notification (\verb|banner motd| command)
\item \textbf{RBAC} (Role-based access control)  restricts user access based on the role of the user. In Cisco IOS, the role-based CLI access feature implements RBAC for router management access. The feature creates different \emph{views} that define which commands are accepted and what configuration information is visible. The central repository server can be an AAA server.
\item Authorize actions
\item Enable management access reporting
\end{itemize}

\textbf{Data plane} security can be implemented using \emph{ACLs}, \emph{antispoofing mechanisms}, and \emph{Layer 2 security} features. 

\begin{itemize}
\item ACLs are used to secure the data plane in a variety of ways: Blocking unwanted traffic or users, Reducing the chance of DoS attacks, Mitigating spoofing attacks, Providing bandwidth control, Classifying traffic to protect the Management and Control planes.
\item Layer 2 security tools are integrated into the Cisco Catalyst switches: Port security, DHCP snooping, Dynamic ARP Inspection (DAI), and IP Source Guard.
\end{itemize}