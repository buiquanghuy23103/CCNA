\chapter{Securing LAN}

\section{Endpoint security}

%There are two internal LAN elements to secure:
%
%\begin{itemize}
%\item \textbf{Endpoints:} Hosts  such as laptops, desktops, servers, and IP phones
%\item \textbf{Network infrastructure:} LAN infrastructure devices interconnect endpoints and typically include switches, wireless devices, and IP telephony devices.
%\end{itemize}
%
%This section focuses on securing endpoints.

New security architectures for the borderless network address these challenges by having endpoints use network scanning elements. Protecting endpoints in a borderless network can be accomplished using the following modern security solutions: Antimalware Protection (AMP), Email Security Appliance (ESA), Web Security Appliances (WSA), Network Admission Control (NAC).\\

\subsection{Anti-malware protection}

Cisco added Sourcefire's Advanced Malware Protection (AMP) technology to protect endpoints and networks more effectively than traditional host-based malware protection. It defeats malware across the extended network before, during, and after an attack.\\

AMP accesses the collective security intelligence of the Cisco Talos Security Intelligence and Research Group (Talos). Talos detects and correlates threats in real time using the largest threat-detection network in the world.\\

AMP protects before, during, and after an attack. AMP is available in a variety of formats:

\begin{itemize}
\item AMP for Endpoints - integrates with Cisco AMP for Networks to deliver comprehensive protection across extended networks and endpoints.
\item AMP for Networks - integrated into dedicated Cisco ASA Firewall and Cisco FirePOWER network security appliances.
\item AMP for Content Security – integrated in Cisco Cloud Web Security or Cisco Web and Email Security Appliances to protect against email and web-based advanced malware attacks.
\end{itemize}

\section{Layer 2 security}