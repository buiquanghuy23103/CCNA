\chapter{Firewall}

\section{ACL}

ACLs can be used to mitigate IP address spoofing and denial of service (DoS) attacks. Use ACL to block inbound packets from the following addresses:

\begin{itemize}
\item All zeros addresses
\item Broadcast addresses
\item Local host addresses (127.0.0.0/8)
\item Reserved private addresses (RFC 1918)
\item IP multicast address range (224.0.0.0/4)
\end{itemize}

Hackers can use ICMP echo packets (pings) to discover network, generate DoS flood attacks, or alter host routing tables. Both ICMP echo and redirect messages should be blocked \emph{inbound} by the router. Several ICMP messages are recommended for proper network operation and should be allowed into the internal network:

\begin{itemize}
\item Echo reply -- Allows users to ping external hosts.
\item Source quench - Requests that the sender decrease the traffic rate of messages.
\item Unreachable - Generated for packets that are administratively denied by an ACL.
\end{itemize}

Several ICMP messages are required for proper network operation and should be allowed to exit the network:

\begin{itemize}
\item Echo -- Allows users to ping external hosts.
\item Parameter problem -- Informs the host of packet header problems.
\item Packet too big -- Enables packet maximum transmission unit (MTU) discovery.
\item Source quench -- Throttles down traffic when necessary.
\end{itemize}

As a rule, block all \emph{other} ICMP message types \emph{outbound}.\\

If SNMP is necessary, exploitation of SNMP vulnerabilities can be mitigated by applying interface ACLs to filter SNMP packets from non-authorized systems.  The most effective means of exploitation prevention is to disable the SNMP server on IOS devices for which it is not required. 

\note See also \emph{CCNA notebook} for ACL configuration and IPv6 ACL.

\section{Firewall}

All firewalls share some common properties: resistant to attacks, the only transit point between networks because all traffic flows through the firewall, enforce the access control policy. There are two configuration models for Cisco IOS Firewall: \hyperref[sec:ClassicFirewall]{Classic Firewall} and \hyperref[sec:ZPF]{Zone-based Policy Firewall (ZPF)}.\\

\subsection{Packet filtering}

\tableStart[\caption{Packet Filtering Firewall Benefits and Limitations}\label{tab:Firewall1}] { |p{5\xm}|p{5\xm}| }
\head{Advantages} & \head{Disadvantages}\w
Simple implementation & Susceptible to IP spoofing \w
Low impact on network performance & Not reliably filter fragmented packets\w
Initial degree of security at the network layer & Use complex ACLs, which can be difficult to implement and maintain\w
Almost all the tasks of a high-end firewall at a much lower cost &  Stateless: examine each packet individually rather than in the context of the state of a connection.\w
\tableEnd

\subsection{Statefull firewall}

\textbf{Stateful firewalls} are the most versatile and the most common firewall technologies in use. Unlike a stateless firewall that uses static packet filtering, stateful filtering tracks each connection and confirms that they are valid. Stateful firewalls use a state table to keep track of the actual communication process. Benefits: prevent spoofing and DoS attacks, provide more stringent control over security. Limitations: cannot prevent Application Layer attacks, does not filter UDP and ICMP packets, cannot track connections that use dynamic port negotiation, not support authentication.

\subsection{Next-Generation Firewall}

Designed with advanced malware protection, the Cisco ASA with FirePOWER services is also called the \textbf{Cisco ASA Next-Generation Firewall} because it is an adaptive, threat-focused firewall. It is designed to provide defense across the entire attack continuum, which includes before, during, and after attacks.

\subsection{DMZ}

A demilitarized zone (DMZ) is a firewall design where there is typically one inside interface connected to the private network, one outside interface connected to the public network, and one DMZ interface, as shown in the figure \ref{DMZfilter}.

\begin{figure}[hbtp]
\caption{DMZ topology and traffic restriction}\label{DMZfilter}
\centering
\includegraphics[width=10\xm]{pictures/DMZfilter.PNG}
\end{figure}

\begin{itemize}
\item Private network $\rightarrow$ DMZ: Inspected 
\item DMZ  $\rightarrow$ Private network: Blocked
\item Public network  $\leftrightarrow$ DMZ: selectively permitted
\item Private network $\rightarrow$ Public network: Inspected
\item Public network $\rightarrow$ Private network: Blocked
\end{itemize}

\subsection{Layered Defense}

A layered defense uses different types of firewalls that are combined in layers. Security policies can be enforced between the layers and inside the layers. A traffic from the untrusted network has to go through the following layers and policies:

\begin{enumerate}
\item Edge router (packet filtering)
\item Bastion host (hardened computer located in the DMZ\footnote{This type of DMZ setup is called a \emph{screened subnet configuration}.}) or Screened firewall
\item Interior screening router
\end{enumerate}

\section{Classic firewall}\label{sec:ClassicFirewall}

\subsection{Introduction}

Classic Firewall (CBAC) is a \emph{stateful} firewall that provides four main functions: traffic filtering, traffic inspection, intrusion detection, and generation of audits and alerts. It can also examine NAT, PAT information, P2P connections. Classic Firewall only provides filtering for those protocols that are specified by an administrator. It can only detects and protects against external attacks that travel through the firewall, but not attacks originating from within the protected network. \\

Classic Firewall creates \emph{temporary} openings in the ACL to allow returning traffic. These entries are created as inspected traffic leaves the network and are removed when the connection terminates or the idle timeout period for the connection is reached. Figure \ref{ClassicFirewall} shows how Classic Firewall inspects SSH traffic.

\begin{figure}[hbtp]
\caption{Classic Firewall inspects SSH traffic}\label{ClassicFirewall}
\centering
\includegraphics[ width=10\xm ]{pictures/ClassicFirewall.PNG}
\end{figure}

\subsection{Configuration}

Take the topology in figure \ref{ClassicConfig} as an example for configuration. Suppose that the administrator wants to allow SSH sessions between the 10.0.0.0 and 172.30.0.0 networks. However, only hosts from the 10.0.0.0 network are allowed to initiate SSH sessions. All other access is denied. 

\begin{figure}[hbtp]
\caption{Network topology}\label{ClassicConfig}
\centering
\includegraphics[width=10\xm]{pictures/ClassicConfig.PNG}
\end{figure}


\begin{sexylisting}{Classic Firewall configuration}
ip access-list extended INSIDE
  permit tcp 10.0.0.0 0.0.0.255 any eq 22
ip access-list extended OUTSIDE
  deny ip any any

ip inspect name FWRULE ssh

interface g0/0
  ip access-group INSIDE in
  ip inspect FWRULE in
interface g0/1
  ip access-group OUTSIDE in
\end{sexylisting}

There are four steps to configure this policy using a Classic Firewall:

\begin{enumerate}
\item \textbf{Define the internal and external interfaces}: G0/0 is the inside interface and G0/1 is the outside interface.
\item \textbf{Configure ACLs for each interface}: The INSIDE ACL allows only SSH traffic from the 10.0.0.0 network; the OUTSIDE ACL will deny inbound traffic from the 172.30.0.0 network.
\item \textbf{Define inspection rules}: The inspection rule FWRULE specifies that traffic will be inspected for SSH connections. This inspection rule has no effect until it is applied to an interface.
\item \textbf{Apply an inspection rule to an interface}: When the FWRULE is applied to inbound traffic on the G0/0 interface, the Classic Firewall configuration will dynamically add an entry to allow inbound SSH traffic from the 172.30.0.0 network. From now on, the FWRULE inspects SSH traffic between 10.0.0.0 and 172.30.0.0 network.
\item \textbf{Verification}: Use \code{show ip inspect sessions} command to verify inspect sessions.
\end{enumerate}

\section{Zone-based Policy Firewall (ZPF)}\label{sec:ZPF}

\subsection{Overview}

ZPFs use the concept of zones to provide additional flexibility. A zone is a group of one or more interfaces that have similar functions or features. By default, the traffic between interfaces in the same zone is not subject to any policy and passes freely. However, all zone-to-zone traffic is blocked. In order to permit traffic between zones, a policy allowing or inspecting traffic must be configured.\\

There are several benefits of a ZPF:

\begin{itemize}
\item Not dependent on ACLs.
\item The router security posture is to block unless explicitly allowed.
\item Policies are easy to read and troubleshoot with the Cisco Common Classification Policy Language (C3PL). C3PL can create traffic policies based on events and affect any given traffic with only one policy, instead of needing multiple ACLs and inspection actions.
\end{itemize}

\subsection{Operation}

The Cisco IOS ZPF can take three possible actions: Inspect, Drop and Pass. ZPF Rules for Transit Traffic depends on the zone that an interface belongs to:

\begin{itemize}
\item Neither intefaces is a zone member: Pass
\item Both interfaces are members of the same zone: Pass
\item Interfaces belong to different zones: Action defined by policy
\item Only one interface is a zone member: Drop
\end{itemize}

The \emph{self zone} is a special zone which is the router itself and includes all the router interface IP addresses. By default, if the router (self zone) is the source or the destination, then all traffic is permitted. The only exception is if the source and destination are a zone-pair with a specific service-policy. In that case, the policy is applied to all traffic.

\subsection{Configuration}

There are five steps to configure a ZPF zone:

\begin{enumerate}
\item Create the zones and Assign zones to appropriate interfaces
\item Identify traffic with class-map
\item Define an action with policy-map
\item Identify a zone-pair and match it to a policy-map
\end{enumerate}

\begin{figure}[hbtp]
\caption{ZPF configuration topology}\label{Zone}
\centering
\includegraphics[scale=0.7]{pictures/Zone.PNG}
\end{figure}


Take the topology in figure \ref{Zone} as an example. 


\begin{sexylisting}{ZPF configuration}
zone security PRIVATE
zone security INTERNET
zone security DMZ
int g0/1
  zone-member security PRIVATE
int s0/0/0
  zone-member security INTERNET    
int g0/0
  zone-member security DMZ
exit

class-map type inspect match-all PRIVATE-ACL-CLASS
  match access-group 100    
class-map type inspect match-any PRIVATE-INTERNET-CLASS
  match protocol http
  match protocol https
  match protocol dns
exit    

policy-map type inspect PRIV-TO-PUB-POLICY
  class type inspect PRIVATE-ACL-CLASS   
  inspect    
  class type inspect PRIVATE-INTERNET-CLASS
  inspect
  class class-default
exit   

zone-pair security PRIVATE-2-INTERNET source PRIVATE destination INTERNET
  service-policy type inspect  PRIV-TO-PUB-POLICY
end

show run | begin class-map
show run | begin class-map
show class-map type inspect
show zone security
show zone-pair security
show policy-map type inspect
show policy-map type inspect zone-pair sessions
\end{sexylisting}

The first step is to create zones and assign them to the appropriate interfaces. Associating a zone to an interface will immediately apply the service-policy that has been associated with the zone. If no service-policy is yet configured for the zone, all transit traffic will be dropped. Use the \code{zone-member security} command to assign a zone to an interface. In the example, g0/1 is assigned the PRIVATE zone, and s0/0/0 is assigned the INTERNET zone, and g0/0 is assigned to DMZ zone.\\

The second step is to use a class-map to identify the traffic. A class is a way of identifying a set of packets based on its contents using \code{match} conditions. Packets must meet one of the match criteria \code{match-any} or all of the match criteria \code{match-all} to be considered a member of the class. Table \ref{tab:classMap} shows the syntax for the \code{class-map} and its sub-commands.\\

\tableStart[\caption{The syntax of class-map command}\label{tab:classMap}] { |p{5\xm}|p{8\xm}| }
\multicolumn{2}{|c|}{ \code{class-map type inspect [match-any | match-all] <class-name>} } \w
\head{Parameter}&\head{Description} \w
\code{match-any} & Packets must meet one of the criteria to be considered a member of the class.\w
\code{match-all} & Packets must meet all of the criteria to be considered a member of the class.\w
\code{match protocol <protocol-name>} & Configure criteria based on specified protocol.\w
\code{match access-group <acl-name>} & Configure criteria based on specified ACL.\w
\code{match class-map <class-name>} & Use another class-map as criteria.\w
\tableEnd

The third step is to assign class-maps (PRIVATE-ACL-CLASS and PRIVATE-INTERNET-CLASS) to a policy-map and define what action (Inspect, Drop, or Pass) should be taken for traffic that is a member of a class. 

\begin{itemize}
\item \code{inspect} -- This action offers state-based traffic control. It tracks UDP or TCP connections and permit the return traffic.
\item \code{drop} -- This is the default action for all traffic. Similar to the implicit deny any at the end of every ACL, , there is an explicit \code{drop} applied to the end of every policy-map.
\item \code{pass} -- This action allows \emph{one-direction} traffic between two zones, and does not track the state of connections. A corresponding policy must be applied to allow return traffic to pass in the opposite direction. This action is ideal for secure protocols, such as IPsec. 
\end{itemize}

The fourth step is to identify a zone pair (PRIVATE-2-INTERNET) using \code{zone-pair security} command, and associate that zone pair to a policy-map (PRIV-TO-PUB-POLICY) using \code{service-policy type inspect} command.\\

The service-policy is now active. HTTP, HTTPS, and DNS traffic sourced from the PRIVATE zone and destined for the PUBLIC zone will be inspected. Traffic sourced from the PUBLIC zone and destined for the PRIVATE zone will only be allowed if it is part of sessions originally initiated by PRIVATE zone hosts.

\subsection{ZPF Configuration Considerations}

\begin{itemize}
\item The router never filters the traffic between interfaces in the same zone.
\item An interface cannot belong to multiple zones.
ZPF can coexist with Classic Firewall although they cannot be used on the same interface. Remove the \code{ip inspect} interface configuration command before applying the \code{zone-member security} command.
\item Traffic can never flow between an interface assigned to a zone and an interface without a zone assignment. Applying the zone-member configuration command always results in a temporary interruption of service until the other zone-member is configured.
\item Communication between zones are, by default, dropped. Unless there exits a service-policy configured for the zone-pair.
\item The \code{zone-member} command does not protect the router itself (traffic to and from the router is not affected) unless the zone-pairs are configured using the predefined self zone.
\end{itemize}