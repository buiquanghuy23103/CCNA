\chapter{Intrusion Prevention System (IPS)}

\section{Introduction}

\subsection{Intrusion Detection Systems (IDSs)}


\textbf{IDS} copies the actual traffic stream and analyzes the copied traffic. IDS works in offline mode, which means several things:

\begin{itemize}
\item IDS passively monitor network traffic without preventing attacks.
\item IDS device is physically positioned in the network so that traffic must be mirrored in order to reach it
\item Network traffic does not pass through the IDS unless it is mirrored
\end{itemize}

\textbf{Advantage:} No impact on the network (delay, jitter). \textbf{Disadvantage:} cannot stop malicious packets, cannot correct tuning required for response action.

\subsection{Intrusion Prevention System (IPS)}

Unlike IDS, IPS is implemented in inline mode. This means that all ingress and egress traffic must flow through it for processing. IPS can detect and immediately solve a network problem. \\

\textbf{Advantages:} stop malicious packets, utilize stream normalization\footnote{a technique used to reconstruct the data stream when the attack occurs over multiple data segments.}. \textbf{Disadvantage:} some impact on network (delay, jitter), IPS overloading or improper configuration negatively affect the network.\\



\subsection{Compare IDS and IPS}
The biggest difference between IDS and IPS is that an IPS responds immediately and does not allow any malicious traffic to pass, whereas no action is taken on malicious packets by the IDS. However, IDS and IPS share several characteristics:

\begin{itemize}
\item Deployed as sensors
\item Use signatures\footnote{A signature is a set of rules that an IDS or IPS uses to detect malicious activity.} to detect patterns in network traffic
\item Can detect atomic signature patterns (single-packet) or composite signature patterns (multi-packet)
\end{itemize}

\subsection{IPS technologies}

There are two primary kinds of IPSs available: host-based and network-based. \\

\textbf{Network-based IPS} devices are implemented at designated network points that enable security managers to monitor network activity while it is occurring, regardless of the location of the attack target. One limitation of them is that they cannot monitor/inspect encrypted packets.\\

\textbf{Host-based IPS (HIPS}) is software installed on a single host to monitor and analyze suspicious activity. Two disadvantages of deploying HIPS are (1) that it cannot create a complete view of the network and (2) every host has to install HIPS.

\section{Signatures}

All signatures are contained in a signature file and uploaded to an IPS on a regular basis. \\

To make the scanning of signatures more efficient, Cisco IOS software relies on signature micro-engines (SMEs), which categorize common signatures in groups. When IDS or IPS is enabled, an SME is loaded to the router. SME then scans atomic and composite packets, then extracts values from those packets and uses them to search for multiple patterns at the same time.\\

Signatures have three distinctive attributes: Type, Trigger (alarm), Action.

\subsection{Type}

Signature types are generally categorized as atomic or composite.\\

An \textbf{atomic signature} can be matched on a single event. Examining theses signatures does not require a state information (stateless) and consumes minimal resources.\\

A \textbf{composite signature} is a stateful signature which identifies a sequence of events. Maintaining state, therefore, is required. The length of time that the signatures must maintain state is known as the event horizon.\\

All signatures are contained in a signature file and uploaded to an IPS on a regular basis.\\

\textbf{SME} (signature micro-engine) categorizes signatures in group and scans for multiple signatures based on group characteristics, instead of one at a time. When IDS or IPS is enabled, an SME is loaded or built on the router. When an SME is built, the router might need to compile the regular expression\footnote{A regular expression is a systematic way to specify a search for a pattern in a series of bytes.} found in a signature. \\

Atomic and composite packets are scanned by the SMEs to recognize the protocols contained in the packets. Then, each SME extracts values from the packet and passes portions of the packet to the regular expression engine. The regular expression engine can search for multiple patterns at the same time. 

\subsection{Trigger}

A signature trigger could be anything that signals an intrusion or security policy violation. There are four types of triggering mechanisms listed below:

\begin{itemize}
\item \textbf{Pattern-based detection} (signature-based detection) compares the network traffic to a database of known attacks.

\item \textbf{Anomaly-based detection} (profile-based detection) involves first defining a profile of what is considered normal traffic, then detecting malicious operations based on this predefined profile. \textbf{Advantage:} unknown attacks can be detected. \textbf{Disadvantage:} Attack activity may be defined as normal traffic.

\item \textbf{Policy-based detection} (behavior-based detection) defines suspicious behaviors based on historical analysis. This technique enables a single signature to cover an entire class of activities without having to specify each individual situation.

\item \textbf{Honey pot-based detection} uses a dummy server to attract attacks. Antivirus and other security vendors tend to use them for research.

\item \textbf{Protocol decodes} breaks down a packet into the fields of a protocol, and then searches for specific patterns in each field. \textbf{Advantage}: more granular inspection of traffic and small number of false positives.
\end{itemize}

Triggering mechanisms mentioned above can generate alarms (table \ref{AlarmType}). A true alarm is the expected outcome. In other words, we would like that intrusion system can generate an alarm in response to known attack traffic (\textbf{true positive}) and be silent when the network is safe (\textbf{true negative}). However, there may be some undesired result. For example, intrusion system fails to generate an alarm after processing attack traffic (\textbf{false negative}), or mistakes normal traffic from malicious and generates alarm (\textbf{false positive}).

\tableStart[\caption{Alarm types}\label{AlarmType}] { |l|l|l|l| }
\head{Alarm type} & \head{Traffic} & \head{Alarm} & \head{Outcome}\w
False Positive & normal &  $\bullet$ & tune alarm\w
False Negative & attack & & tune alarm\w
True Positive & attack & $\bullet$ & ideal setting\w
True Negative & normal & & ideal setting\w
\tableEnd

\subsection{Actions}

When a signature detects the activity for which it is configured, the signature triggers one or more actions. 

\paragraph{Generate an alert.} There are two categories of alert: atomic and summary alert. Atomic alerts are generated every time a signature triggers. A summary alert is a single alert that indicates multiple occurrences of the same signature from the same source address or port. 

\paragraph{Logging:} this activity may start on packets that contain attacker's address (logging attacker packets), or attacker-victim address pair (logging pair packet), or victim's address (logging victim packet).

\paragraph{Deny (drop):} this action can be expanded to drop all packets for a specific connection or even all packets from a specific host for a certain amount of time. By dropping traffic for a connection or host, the IPS conserves resources without having to analyze each packet separately.

\paragraph{Resetting a TCP connection:} Many IPS devices use the TCP reset action to abruptly end a TCP connection that is performing unwanted operations. 

\paragraph{Blocking future activity} is the action that updates ACL on \emph{one} of the infrastructure devices. After a configured period of time, the IPS device removes the ACL. This action allows a single IPS device can stop traffic at multiple locations throughout the network, regardless of the location of the IPS device. 

\paragraph{Allowing the Activity:} this action is necessary so that an administrator can define exceptions to configured signatures. For example, suppose that the IT department routinely scans its network to find security hole. 


\section{Monitoring and management} 

\subsection{Monitoring strategy}

There are four factors to consider when implementing a monitoring strategy: Management method, Event correlation, Security staff, and Incident response plan. Event correlation refers to the process of correlating events happening at different points across a network based on time stamp. \\

There are three GUI-based IPS device managers available: Cisco Configuration Professional, Cisco IPS Manager Express (IME), Cisco Security Manager.\\

Alarms are generated when an enabled signature is triggered. These alarms are stored on the sensor and sent to the administrator using either Syslog or Secure Device Event Exchange (SDEE) protocol. A message generated by SDEE protocol will look like as follow:

\begin{verbatim}
%IPS-4-SIGNATURE:Sig:1107 Subsig:0 Sev:2 RFC1918 address [192.168.121.1:137 ->192.168.121.255:137] 
\end{verbatim}

\subsection{Global Correlation and SensorBase Network}

Cisco Global Correlation provides regular threat updates to IPS sensors from a database called the Cisco SensorBase Network, which has information about IP addresses with a reputation. The sensor uses this information to determine harmful traffic. \\

Communication between sensors and the SensorBase Network involves an HTTPS request and response over TCP/IP. There are three modes for sensor to participate SensorBase Network: off, partial participation, and full participation.\\

The SensorBase Network is part of a Cisco Security Intelligence Operation (SIO), a security intelligence ecosystem that baselines the current state of threats on a worldwide basis. SIO provides live threat prevention based on malware outbreaks, current vulnerabilities, and zero-day attacks.


\section{IPS configuration on CLI}

\subsubsection{Preparation}

Prior to configuring IPS, it is necessary to download the signature files \verb|IOS-Sxxx-CLI.pkg|, and a public crypto key \verb|realm-cisco.pub.key.txt| from cisco.com. Only registered customers can download the package files and key.\\

Next, we use \code{mkdir <dir-name>} command to create a directory in flash, for example \verb|flash:IPS|, to store the signature file. After that, we paste the content of \verb|realm-cisco.pub.key.txt| file to the router at the global configuration prompt. This issues the various commands to generate the RSA key. If the key is configured incorrectly, an error message is generated as follow

\begin{verbatim}
%IPS-3-INVALID_DIGITAL_SIGNATURE: Invalid Digital Signature found (key not found)
\end{verbatim}

In such case, the key must be removed and then reconfigured. Use the \code{no crypto key pubkey-chain rsa} and the \code{no named-key realm-cisco.pub signature} commands. Then repeat the procedure in Step 3 to reconfigure the key.\\

If you want to use SDEE for logging notification, the HTTP or HTTPS server must first be enabled (\code{ip http server} or \code{ip https server} command), then use the \code{ip ips notify sdee} command. The buffer size can be altered with the \code{ip sdee events} command. The \code{clear ip ips sdee} command clears SDEE events or subscriptions. The number of events stored in RAM can be altered with the \code{ip sdee events <num>} command. If you prefer traditional syslog, use the \code{ip ips notify log} command.\\

The location of signature files is required so that IPS knows where to search for threat patterns. IPS configuration first involves creating a rule associated with an optional access list. All traffic that is permitted by the ACL is subject to inspection by the IPS.

\begin{sexylisting}{IPS preparation}
mkdir flash:IPS
no crypto key pubkey-chain rsa
no named-key realm-cisco.pub signature

<Script from realm-cisco.pub.key.txt>

ip https server
ip ips notify sdee
ip ips config location flash:IPS
ip ips name IOSIPS list ALLOW-HTTP
\end{sexylisting}

\subsubsection{Signature category}

All signatures are grouped into three categories: \verb|all|, \verb|basic|, and \verb|advanced|. These signatures can be retired or unretired. \textbf{Retiring} a signature means that IPS does not compile that signature into memory for scanning. \textbf{Un-retiring} a signature instructs IPS to compile the signature into memory and use it to scan traffic.\\

Do not unretire the \verb|all| category, because IPS cannot compile and use all the signatures at one time because it will run out of memory. Instead, you should retire \verb|all| category before selecting any desired category.\\

IPS processes the signature category commands in the order listed in the configuration. If multiple categories are configured and a signature belongs to more than one of them, only the last configured category matters.

\begin{sexylisting}{IPS category}
ip ips signature-category
	category all
	retired true
	exit
  category ios_ips basic
  retired false
  exit
exit  

interface g0/0
  ip ips IOSIPS in
interface g0/1
  ip ips IOSIPS in
  ip ips IOSIPS out
  end
show ip ips all
show ip ips signature count  
\end{sexylisting}

\subsection{Modification}
The command \code{ip ips signature-definition} is used to modify a specific signature, including retire/unretire and event action.

\begin{sexylisting}{Retire a specific signature}
ip ips signature-definition
  signature 6130 10
  status
    retired true
end 
\end{sexylisting}

\begin{sexylisting}{Change actions of a signature}
ip ips signature-definition
  signature 6130 10
  engine
    event-action produce alert
    event-action deny-packet-inline
    event-action reset-tcp-connection
end    
\end{sexylisting}

\begin{sexylisting}{Change actions of a signature category}
ip ips signature-definition
  category ios_ips basic
    event-action produce alert
    event-action deny-packet-inline
    event-action reset-tcp-connection
end  
\end{sexylisting}

\tableStart[\caption{Parameters of event-action command}\label{event-action}] {|l|p{8\xm}|}
\verb|deny-attacker-inline| & Terminates the current and future packets from a particular attacker address for a period of time\w
\verb|deny-connection-inline| & Terminates packets come on this TCP flow.\w
\verb|deny-packet-inline| & Terminates the packet.\w
\verb|produce-alert| & Writes event to Event Store as an alert.\w
\verb|reset-tcp-connection| & Send TCP reset signal and terminate the TCP flow. Only works on TCP signatures that analyze a single connection. Not work for sweeps or floors.\w
\tableEnd

