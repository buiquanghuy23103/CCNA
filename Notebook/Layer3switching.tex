\chapter{Layer 3 Switching}
Inter-VLAN routing using the router-on-a-stick method was simple to implement because routers were usually available in every network. However, most modern enterprise networks use multilayer switches to achieve high-packet processing rates using hardware-based switching. Layer 3 switches usually have packet-switching throughputs in the millions of packets per second (pps), whereas traditional routers provide packet switching in the range of 100,000 pps to more than 1 million pps.\par 
Many users are in separate VLANs, and each VLAN is usually a separate subnet. Therefore, it is logical to configure the distribution switches as Layer 3 gateways for the users of each access switch VLAN. This implies that each distribution switch must have IP addresses matching each access switch VLAN. This can be achieved by using Switch Virtual Interfaces (SVIs) and routed ports.
\begin{itemize}
\item \textbf{Routed port} -- A pure Layer 3 interface similar to a physical interface on a Cisco IOS router.
\item \textbf{Switch virtual interface (SVI)} -- A virtual VLAN interface for inter-VLAN routing. In other words, SVIs are the virtual-routed VLAN interfaces.
\end{itemize}
A routed port is a physical port that acts similarly to an interface on a router. Unlike an access port, a routed port is not associated with a particular VLAN. A routed port behaves like a regular router interface. Unlike Cisco IOS routers, routed ports on a Cisco IOS switch do not support subinterfaces. Routed ports are used for point-to-point links.  In a switched network, routed ports are mostly configured between switches in the core and distribution layer. \par 
An SVI is a virtual interface that is configured for each VLAN that exists on the switch. It is considered to be virtual because there is no physical port dedicated to the interface. It can perform the same functions for the VLAN as a router interface would, and can be configured in much the same way as a router interface (i.e., IP address, inbound/outbound ACLs, etc.).