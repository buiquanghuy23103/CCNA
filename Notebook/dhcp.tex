\chapter{DHCPv4}

\section{Operation}

DHCPv4 works in a client/server mode. When a client communicates with a DHCPv4 server, the server assigns or leases an IPv4 address to that client. The client connects to the network with that leased IP address until the lease expires.\\

The client must contact the DHCP server periodically to extend the lease. This lease mechanism ensures that clients that move or power off do not keep addresses that they no longer need. When a lease expires, the DHCP server returns the address to the pool where it can be reallocated as necessary.\\

\subsection{Lease origination}

When the client boots (or otherwise wants to join a network), it begins DORA\footnote{DORA = Discovery, Offer, Request, Acknowledgement} process to obtain a lease.

\begin{enumerate}
\item A client starts the process with a broadcast \textbf{DHCP-DISCOVER} message to finds available DHCPv4 servers.

\item When the DHCPv4 server receives a DHCP-DISCOVER message, it reserves an available IPv4 address to lease to the client. It then sends a \textbf{DHCP-OFFER} message to the requesting client. The server also creates an ARP entry consisting of the MAC address of the requesting client and the leased IPv4 address of the client.

\item When the client receives the DHCP-OFFER from the server, it broadcasts\textbf{DHCP-REQUEST} messages. The DHCP-REQUEST serves as an acceptance notice to the selected server and an implicit decline to any other servers.

\item On receiving the DHCP-REQUEST message, the server verifies the lease information with an ICMP ping to that address to ensure it is not being used already. If there is \emph{no} ICMP echo reply, then the address is is not being used by any client. Otherwise, that address is being used and the server has to send DHCP-OFFER again.

\item DHCP server sends  unicast DHCP-ACK message to the client. The DHCP-ACK message informs the client that the IP address is valid. Because The DHCP-ACK message is a duplicate of the DHCP-OFFER, it also provides IP information for the client.

\item When the client receives the DHCPACK message, performs an ARP lookup for the assigned address.
If there is no reply to the ARP, the client knows that the IPv4 address is valid and starts using it as its own.
\end{enumerate}



\subsection{Lease renewal}

\begin{enumerate}
\item Before the lease expires, the client sends a DHCPREQUEST message directly to DHCPv4 server. If a DHCPACK is not received within a specified amount of time, the client broadcasts another DHCPREQUEST so that one of the other DHCPv4 servers can extend the lease.

\item On receiving the DHCPREQUEST message, the server verifies the lease information by returning a DHCPACK
\end{enumerate}

\subsection{Relay agent}

Sometimes, network clients are not on the same subnet as DHCP servers. Because routers do not forward broadcasts, the DHCP-REQUEST from clients are not sent to DHCP server.\\

Cisco offers a solution called Cisco IOS helper address. This solution enables a router to forward DHCP-REQUEST broadcasts to the DHCPv4 server. When
a router forwards address assignment/parameter requests, it is acting as a DHCPv4 \emph{relay agent}.

\section{Message}

DHCPv4 messages UDP encapsulation. The server uses port 67, the client uses port 68.

\subsection{Message format}

\paragraph{Operation (OP) code}specifies general type of message: request (1), reply (2).

\begin{figure}[hbtp]
\caption{DHCPv4 message format}
\centering
\includegraphics[ width=0.8\textwidth ]{pictures/DHCPmessage.PNG}
\end{figure}

\paragraph{Hardware type} Ethernet (1), Frame Relay (15), Serial (20)

\paragraph{Hop}Controls the forwarding of messages. Set to 0 by a client before transmitting a request.

\paragraph{Transaction identifier} used by client to match the request with replies from DHCPv4 server.

\paragraph{Seconds} amount of time (in seconds) elapsed since a client attempted to acquire or renew a lease.

\paragraph{Flag}Used by a client that does not know its IPv4 address when it sends a
request. Only one of the 16 bits—the broadcast flag—is used. A value of 1 in
this field tells the DHCPv4 server or relay agent receiving the request that the
reply should be sent as a broadcast.

\subsection{DHCP-DISCOVER}

When the client boots, it has no way of knowing the subnet to which it belongs. Therefore, destination
IPv4 address of DCHP-DISCOVER is 255.255.255.255, the destination MAC address is FF:FF:FF:FF:FF:FF. The source MAC address is the MAC address of the client. The client does not have a configured IPv4 address
yet, so the source IPv4 address is 0.0.0.0 

\subsection{DHCP-OFFER}

DHCP-OFFER contains initial configuration information
for the client: IPv4 address for client, subnet mask, lease duration, and IPv4 address of the DHCPv4 server. The DHCPOFFER message can be configured to include other information, such as the lease renewal time and DNS address.


\section{Configuration}

\subsection{DHCPv4 server}

\begin{description}
\item[Step 1. Exclude addresses]Some IPv4 addresses in a pool are assigned to network devices that require static address assignments. Therefore, these IPv4 addresses should not be assigned to other devices.

\begin{verbatim}
R1(config)# ip dhcp excluded-address <ip-address>
R1(config)# ip dhcp excluded-address 192.168.10.1

R1(config)# ip dhcp excluded-address <range-of-address>
R1(config)# ip dhcp excluded-address 192.168.10.1 192.168.10.9
\end{verbatim}

\item[Step 2. Configure address pool]Define a pool of addresses to assign to clients.

\begin{verbatim}
R1(config)# ip dhcp pool <pool-name>
R1(dhcp-config)# network <network-range>

R1(config)# ip dhcp pool LAN-POOL-1
R1(dhcp-config)# network 192.168.10.0 255.255.255.0
\end{verbatim}

\item[Step 3. Default gateway]Define the default gateway router for clients. 

\begin{verbatim}
R1(dhcp-config)# default-router 192.168.10.1
\end{verbatim}

\item[Step 4. Relay agent]If network clients are not on the same subnet as DHCP servers, configure the default gateway as a relay agent.

\begin{verbatim}
R1(config)# interface g0/0
R1(config-if)# ip helper-address 192.168.11.6
\end{verbatim}

\item[Step 4 (Optional). Other DHCP specifics]

\begin{verbatim}
R1(dhcp-config)# dns-server 192.168.11.5
R1(dhcp-config)# domain-name example.com
\end{verbatim}

\item[Step 5. Verification] The \verb|show ip dhcp binding| command displays a list of all IPv4 address to MACaddress bindings that have been provided by the DHCPv4 service. The \verb|show ip dhcp server statistics| command verifies that messages are being
received or sent by the router. This command displays count information regarding the number of DHCPv4 messages that have been sent and received.

\begin{verbatim}
R1# show run | sec dhcp
R1# show ip dhcp binding
R1# show ip dhcp statistics
\end{verbatim}
\end{description}