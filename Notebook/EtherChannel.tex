\chapter{EtherChannel}

\section{Advantages}

\begin{itemize}
\item Most configuration tasks can be done on the EtherChannel interface instead of each individual port, ensuring configuration consistency throughout the links.
\item EtherChannel creates an aggregation that is seen as one logical link. 
\item EtherChannel provides redundancy.
\item Load balancing takes place between links that are part of the same EtherChannel. 
\item EtherChannel relies on existing switch ports. There is no need to upgrade the link to a faster and more expensive connection to have more bandwidth.
\end{itemize}

\section{Port Aggregation Protocol (PagP)}

PAgP (pronounced ``Pag – P”) is a Cisco-proprietary protocol that aids in the  creation of EtherChannel links. PAgP helps create the EtherChannel link by detecting the configuration of each side and ensuring that links are compatible so that the EtherChannel link can be enabled when needed. PAgP can be configured in one of three models:

\begin{itemize}
\item \textbf{On} -- This mode forces the interface to channel without PAgP. Interfaces configured in the on mode do not exchange PAgP packets.
\item \textbf{PAgP Active} -- This PAgP mode places an interface in an active negotiating state in which the interface initiates negotiations with other interfaces by sending PAgP packets.
\item \textbf{PAgP Passive} -- This PAgP mode places an interface in a passive negotiating state in which the interface responds to the PAgP packets that it receives, but does not initiate PAgP negotiation.
\end{itemize}

The modes must be compatible on each side as shown in table \ref{PAgP-mode}.

\begin{table}[htbp]
\caption{PAgP Establishment}
\label{PAgP-mode}
\begin{tabular}{|l|l|l|}
\hline
S1             & S2                & EtherChannel establishment \\ \hline
Active      & Passive/Active    & Yes                        \\ \hline
On             & On                & Yes                        \\ \hline
Passive           & Passive/On           & No                         \\ \hline
Not configured & Passive/Active/On & No                         \\ \hline
Active      & on                & No                         \\ \hline
\end{tabular}
\end{table}

\section{Link Aggregation Control Protocol (LACP)}

LACP is part of an 802.3ad that aids in the  creation of EtherChannel links. Because LACP is an IEEE standard, it can be used to facilitate EtherChannels in multivendor environments, including Cisco devices. LACP allows for eight active links, and also eight standby links. A standby link will become active should one of the current active links fail. PAgP can be configured in one of three models:

\begin{itemize}
\item \textbf{On} -- This mode forces the interface to channel without LACP. Interfaces configured in the on mode do not exchange LACP packets.
\item \textbf{LACP active} -- Similar to PAgP Active mode.
\item \textbf{LACP passive} -- Similar to PAgP Passive mode. negotiation.
\end{itemize}

The modes must be compatible on each side as shown in table \ref{PAgP-mode}.
\begin{table}[htbp]
\caption{LACP Establishment}
\label{LACP-mode}
\begin{tabular}{|l|l|l|}
\hline
S1             & S2                & EtherChannel establishment \\ \hline
Active      & Passive/Active    & Yes                        \\ \hline
On             & On                & Yes                        \\ \hline
Passive           & Passive/On           & No                         \\ \hline
Not configured & Passive/Active/On & No                         \\ \hline
Active      & on                & No                         \\ \hline
\end{tabular}
\end{table}

\section{Configuration}

\subsection{Implementation restrictions}

The EtherChannel provides full-duplex bandwidth between one switch and another switch or host. Currently each EtherChannel can consist of up to eight compatibly-configured Ethernet ports. However, interface types cannot be mixed. For example, Fast Ethernet and Gigabit Ethernet cannot be mixed within a single EtherChannel.\\

EtherChannel creates a one-to-one relationship; that is, one EtherChannel link connects only two devices. The individual EtherChannel group member port configuration must be consistent on both devices: 

\begin{itemize}
\item \textbf{EtherChannel support:} All Ethernet interfaces on all modules must support EtherChannel with no requirement that interfaces be physically contiguous, or on the same module.

\item \textbf{Speed and duplex:} Configure all interfaces in an EtherChannel to operate at the same speed and in the same duplex mode.

\item \textbf{VLAN match:} All interfaces in the EtherChannel bundle must be assigned to the same VLAN. If the physical ports of one side are configured as trunks, the physical ports of the other side must also be configured as trunks within the same native VLAN.

\item \textbf{Range of VLANs:} An EtherChannel supports the same allowed range of VLANs on all the interfaces in a trunking EtherChannel. If the allowed range of VLANs is not the same, the interfaces do not form an EtherChannel, even when set to auto or desirable mode.
\end{itemize}

\subsection{Configuration}

\paragraph{Step 1:} Specify the interfaces that compose the EtherChannel group using the \verb|interface range <interface>| global configuration mode command. A good practice is to start by shutting down those interfaces, so that any incomplete configuration does not create activity on the link. At the end of this step, make sure that none of restrictions above are broken.

\paragraph{Step 2:} Create the port channel interface with a channel group number. The mode active keywords identify this as an LACP EtherChannel configuration.

\begin{verbatim}
S1(config)# interface range f0/1-2
S1(config-if-range)# shutdown
S1(config-if-range)# channel-group 1 mode active
S1(config-if-range)# no shutdown
\end{verbatim}

\paragraph{Step 3:} Change layer 2 settings on port channel interface, so that two sides of the EtherChannel link have the same configuration.

\begin{verbatim}
S1(config)# interface port-channel 1
S1(config-if)# no shutdown
S1(config-if)# switchport mode trunk
S1(config-if)# switchport trunk allowed vlan 1,10,20,99
\end{verbatim}