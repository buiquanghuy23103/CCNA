\chapter{EtherChannel}

\section{Introduction}

EtherChannel creates an aggregation that is seen as one logical link, which provides redundancy and load balancing. Most configuration tasks can be done on the EtherChannel interface instead of each individual port, ensuring configuration consistency throughout the links. EtherChannel implementation is cost effective, because this protocol relies on existing switch ports, which means there is no need to upgrade the link to a faster and more expensive connection to have more bandwidth.\\

\textbf{PAgP} (pronounced ``Pag – P”) is a \textbf{Cisco-proprietary} protocol that aids in the creation of EtherChannel links. PAgP helps create the EtherChannel link by detecting the configuration of each side and ensuring that links are compatible so that the EtherChannel link can be enabled when needed. PAgP can be configured in one of three models:

\begin{itemize}
\item \textbf{On} -- This mode forces the interface to channel without PAgP. Interfaces configured in this mode do not exchange PAgP packets.
\item \textbf{PAgP Desirable} -- The interface in this mode \emph{initiates negotiations} with other interfaces by sending PAgP packets.
\item \textbf{PAgP Auto} -- The interface in this mode only responds to the PAgP packets that it receives, but does not initiate PAgP negotiation.
\end{itemize}

The modes must be compatible on each side as shown in table \ref{PAgP-mode}. Otherwise, the status of port-channel will become \verb|err-disabled|.


\tableStart[\caption{PAgP Establishment}\label{PAgP-mode}] {|l|l|l|}
\head{S1}             & \head{S2}                & \head{EtherChannel establishment} \w
Desirable      & Auto/Desirable    & Yes                        \w
On             & On                & Yes                        \w
Auto           & Auto/On           & No                         \w
Not configured & Auto/Desirable/On & No                         \w
Desirable      & on                & No                         \w
\tableEnd

\textbf{LACP} is a \textbf{standard-based} protocol and is a part of an \emph{802.3ad} that aids in the  creation of EtherChannel links. Because LACP is an IEEE standard, it can be used to facilitate EtherChannels in \emph{multivendor} environments, including Cisco devices. LACP allows for eight active links, and also eight standby links. A standby link will become active should one of the current active links fail. PAgP can be configured in one of three models:

\begin{itemize}
\item \textbf{On} -- This mode forces the interface to channel without LACP. Interfaces configured in this mode do not exchange LACP packets.
\item \textbf{LACP Active} -- The interface in this mode \emph{initiates negotiations} with other interfaces by sending LACP packets.
\item \textbf{LACP Passive} -- The interface in this mode only responds to the LACP packets that it receives, but does not initiate  LACP.
\end{itemize}

The modes must be compatible on each side as shown in table \ref{LACP-mode}.

\tableStart[\caption{LACP Establishment}\label{LACP-mode}] {|l|l|l|}
\head{S1}             & \head{S2}                & \head{EtherChannel establishment} \w
Active      & Passive/Active    & Yes                        \w
On             & On                & Yes                        \w
Passive           & Passive/On           & No                         \w
Not configured & Passive/Active/On & No                         \w
Active      & on                & No                         \w
\tableEnd

\section{Configuration}

\subsection{Restrictions}

The EtherChannel provides full-duplex bandwidth between one switch and another switch or host. Currently each EtherChannel can consist of up to eight compatibly-configured Ethernet ports. However, interface types cannot be mixed. For example, Fast Ethernet and Gigabit Ethernet cannot be mixed within a single EtherChannel.\\

EtherChannel creates a one-to-one relationship; that is, one EtherChannel link connects only two devices. The individual EtherChannel group member port configuration must be consistent on both devices: 

\begin{itemize}
\item \textbf{EtherChannel support:} All Ethernet interfaces on all modules must support EtherChannel with no requirement that interfaces be physically contiguous, or on the same module.

\item \textbf{Speed and duplex:} Configure all interfaces in an EtherChannel to operate at the same speed and in the same duplex mode.

\item \textbf{VLAN match:} All interfaces in the EtherChannel bundle must be assigned to the same VLAN. If the physical ports of one side are configured as trunks, the physical ports of the other side must also be configured as trunks within the same native VLAN.

\item \textbf{Range of VLANs:} An EtherChannel supports the same allowed range of VLANs on all the interfaces in a trunking EtherChannel. If the allowed range of VLANs is not the same, the interfaces do not form an EtherChannel, even when set to auto or desirable mode.
\end{itemize}

\subsection{Configuration}

\begin{enumerate}
\item Specify the interfaces that compose the EtherChannel group using the \verb|interface range <interface>| global configuration mode command. A good practice is to start by shutting down those interfaces, so that any incomplete configuration does not create activity on the link. At the end of this step, make sure that none of restrictions above are broken.

\item Create the port channel interface with a channel group number. The mode active keywords identify this as an LACP EtherChannel configuration.

\item Change layer 2 settings on port channel interface, so that two sides of the EtherChannel link have the same configuration.
\end{enumerate}

\begin{sexylisting}{EtherChannel}
interface range f0/1-2
  shutdown
  channel-group 1 mode active
  no shutdown
interface port-channel 1
  no shutdown
  switchport mode trunk
  switchport trunk allowed vlan 1,10,20,99  
\end{sexylisting}

