\section{Single-area vs multiarea OSPF }
Single-area OSPF is useful in smaller networks, however, if an area becomes too big, the following issues must be addressed:
\begin{description}
	\item[Large routing table] Route summarization is not performed by default, therefore, the routing table can become very large.
	\item[Large link-state database(LSDB)] The LSDB describes the topology of the entire network. The bigger the network, the larger the LSDB.
	\item[Frequent SPF algorithm calculations] Even small changes in a specific area of the network makes the routers recalculate the SPF algorithm and update the LSDB for the entire routing domain.
	\end{description}
To make OSPF more efficient and scalable, Multiarea OSPF is implemented in a two-layer area hierarchy:
\begin{description}
	\item[Backbone (Transit) area] A backbone area directly connected with all other areas. All traffic moving from one area to another area must traverse the backbone area. Generally, end users are not found within a backbone area. The backbone area is also called OSPF area 0.
	\item[Regular (Non-backbone) area)] Connects users and resources. By default, a regular area does not allow traffic from another area to use its links to reach other areas. All traffic from other areas must cross a transit area.
	\end{description}
The optimal number of routers per area varies based on factors such as network stability, but Cisco recommends the following guidelines:
\begin{itemize}
	\item An area should have no more than 50 routers.
	\item A router should not be in more than three areas.
	\item Any single router should not have more than 60 neighbors.
	\end{itemize}
Multiarea OSPF have these advantages:
\begin{description}
	\item[Smaller routing table] There are fewer routing table entries as network addresses can be summarized between areas.
	\item[Reduced link-state update overhead] Fewer routers exchanging LSAs because LSA flooding stops at the area boundary.
	\item[Reduced frequency of SPF calculations] Routing still occurs between the areas (interarea routing). However, the CPU intensive routing operation of recalculating the SPF algorithm is done only for routes within an area.
	\end{description}